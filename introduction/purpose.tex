\section{目的}
本研究では, 設計に資する収穫物周辺の特性を定量化し, エンドエフェクタの設計指標を構築することを試みる.
また, 3Dスキャナを用いて収穫物の3Dモデル化を行い, 3DCGソフトウェアであるBlenderを用いて特性を計測する.
そして, 従来から提案されているエンドエフェクタを参考に簡易的なエンドエフェクタのモデルを作成し, 実環境でエンドエフェクタモデルを収穫物にアプローチさせることで, 設計指標からの成功率と実環境での成功率を比較し, 構築した設計指標の妥当性を検証する.
