%!TEX root = ../thesis.tex

\section{背景}
農林水産省によると, 日本の農業従事者数は, 2000年から2023年にかけて約50%減少している.
また, 2023年の農業従事者のうち約7割が65歳以上の高齢者となっている.
そのため, 日本の農業分野では人手不足と高齢化による農作業の負担増大が深刻な問題となってきている.
これらの問題を解決するために作物を自動で収穫することができるロボットの開発が望まれる.
収穫ロボットに求められる技術のうちの1つに収穫用のエンドエフェクタが挙げられる.

現在, 収穫用エンドエフェクタに関する研究は多く, 既に様々なエンドエフェクタが提案されている.
%果実をグリッパで把持して花柄を刃で切断するものや[],
しかし, それらのエンドエフェクタの設計指針については明確に示されていない.
そのため, 収穫する作物に対して優れた形状がわからないことや, 具体的な改善点が見つけにくいことが問題となる.

Fig.?のような場合において, 赤で示されたピーマンは花柄が露出しているが果実は葉で隠れてしまっているのに対し, オレンジで示されたピーマンは花柄が葉で隠れているが果実は露出している.
赤で示されたピーマンは花柄にアプローチしやすく, オレンジで示されたピーマンは果実にアプローチしやすことがわかる.
このことから, 収穫物のいくつかの特性を定量化することで収穫物に適したエンドエフェクタの形を考える.

収穫物は多種多様であり, 一般論で議論することは難しいため,以後の議論はピーマンを対象に行う.


\subsection{RoboCup}

\begin{figure}[hbtp]
  \centering
 \includegraphics[keepaspectratio, scale=0.8]
      {images/RaspberryPiMouse.png}
 \caption{Example}
 \label{Fig:Example}
\end{figure}

\subsubsection{etc...}
\newpage
