\section{実験目的}
構築した設計指標の有効性を検証するため, 従来の提案されているエンドエフェクタを参考に動力のない簡易的なモデルを3つ作成し, 前章で3Dスキャンした30個のピーマンに対してそれぞれアプローチさせて, その成功率を求める.
そして, 設計指標から算出した成功率と実環境での成功率を比較する.
簡易モデルを作成する際に参考にするエンドエフェクタは, Bacらが提案したフィンレイエンドエフェクタ, Aradらが提案した振動ナイフのエンドエフェクタ, AGRISTが開発した「L」に取り付けられているエンドエフェクタの3つである.
これらはどれもピーマンまたはパプリカを収穫するエンドエフェクタであり, その中で引用件数が比較的多い, または市販されているものである.
