\section{設計に資する収穫物の特性}
収穫物のどの部分に接近しやすいかで好ましいエンドエフェクタの形状が決まるため, 個体の特性よりも株の特性から議論する.
アプローチの容易さを推測する特性としては以下が考えられる.

\begin{center}
\begin{tabular}{|lr|} \hline
  花柄露出面積 & 露出している花柄の投影面積\\ \hline
  花柄側面障害物距離 & 花柄の横方向にある障害物の距離\\ \hline
  果実露出面積 & 露出している果実の投影面積\\ \hline
  果実側面障害物距離 & 果実の横方向にある障害物の距離\\ \hline
  果実上面障害物距離 & 果実の上方向にある障害物の距離\\ \hline
  果実下面障害物距離 & 果実の下方向にある障害物の距離\\ \hline
\end{tabular}
\end{center}