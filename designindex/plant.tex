\section{設計に資する収穫物の特性}
収穫物のどの部分に接近しやすいかで好ましいエンドエフェクタの形状が決まるため, まずは, 株の特性について議論する.
アプローチの容易さを推測する特性としては以下が考えられる.

\begin{itemize}
  \item 花柄露出面積$S_p$: 露出している花柄の投影面積
  \item 花柄側面障害物間距離$d_p$: 花柄の横方向にある障害物の最小距離
  \item 果実露出面積$S_f$: 露出している果実の投影面積
  \item 果実側面障害物間距離$d_fside$: 果実の横方向にある障害物の最小距離
  \item 果実上面障害物間距離$d_fabove$: 果実の上方向にある障害物の最小距離
  \item 果実下面障害物間距離$d_fbeside$: 果実の下方向にある障害物の最小距離
\end{itemize}
