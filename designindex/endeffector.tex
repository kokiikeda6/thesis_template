\section{エンドエフェクタの形状}
3.1節を踏まえて, アプローチの容易さを推測するために必要なエンドエフェクタの形状は以下が考えられる.

\begin{itemize}
  \item 花柄アプローチ幅$l_p$: 花柄にアプローチする部分の幅
  \item 花柄アプローチ高さ$h_p$: 花柄に触れる部分の高さ
  \item 花柄アプローチ面積$S_papproach$: 把持または切断時に予想されるアプローチ部分の面積, $h_p$ \verb|×| (花柄の幅の平均)で算出する
  \item 果実アプローチ幅$l_fside$, $l_fabove$, $l_funder$: 果実にアプローチする部分の幅, 幅の方向はエンドエフェクタによって異なる
  \item 果実アプローチ高さ$h_p$: 果実に触れる部分の高さ
  \item 果実アプローチ面積$S_fapproach$: 把持時に予想されるアプローチ部分の面積, $h_f$ \verb|×| (果実の幅の平均)で算出する
\end{itemize}