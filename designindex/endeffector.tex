\section{エンドエフェクタのパラメータ}
3.1節を踏まえて, アプローチの容易さを推測するために必要なエンドエフェクタのパラメータは以下が考えられる.

\begin{itemize}
  \item 花柄アプローチ幅$l_p$: 花柄にアプローチする部分の幅
  \item 花柄アプローチ高さ$h_p$: 花柄に触れる部分の高さ
  \item 花柄アプローチ面積$S_papproach$: 把持または切断時に予想されるアプローチ部分の面積, $h_p$ \verb|×| (花柄の幅の平均)で算出する
  \item 果実アプローチ幅$l_fside$, $l_fabove$, $l_fbelow$: 果実にアプローチする部分の幅, 幅の方向はエンドエフェクタによって異なる
  \item 果実アプローチ高さ$h_p$: 果実に触れる部分の高さ
  \item 果実アプローチ面積$S_fapproach$: 把持時に予想されるアプローチ部分の面積, $h_f$ \verb|×| (果実の幅の平均)で算出する
\end{itemize}