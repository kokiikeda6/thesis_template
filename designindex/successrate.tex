\section{成功率の算出}
各エンドエフェクタはそれぞれピーマンにアプローチする部分が異なるため, 必要に応じて指標を取り入れることになる.
例えば, AGRISTのLに搭載されているエンドエフェクタの場合は, 果実にアプローチを行わないため, 花柄に関わる指標しか取り入れないこととなる.
取り入れた指標をすべて満たすものを成功とし, それ以外は失敗とする.
成功率$p_h$は以下のように表す.

\vspace{10mm}
成功率$p_h$ = 指標をすべて満たすもの / 計測した収穫物の数
\vspace{10mm}

AGRISTのLに搭載されているエンドエフェクタの場合, 成功率に関わる設計指標は, 花柄側面障害物間距離が許容範囲内かというのと花柄アプローチ面積よりも花柄が露出しているかの2つである.
よって成功率は以下のようになる.

\vspace{10mm}
$p_h = p(d_p > l_p \cap S_p > S_papproach)$
\vspace{10mm}
