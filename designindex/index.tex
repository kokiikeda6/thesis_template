\section{設計指標}
収穫物の特性とエンドエフェクタの形状から設計指標を構築する.
設計指標は以下のように設定した.

\begin{itemize}
  \item 花柄側面障害物間距離が許容範囲内か: 花柄側面障害物距離 \verb|>| 花柄アプローチ幅
  \item 花柄アプローチ面積より花柄が露出しているか: 花柄露出面積 \verb|>| 花柄アプローチ面積
  \item 果実側面障害物間距離が許容範囲内か: 果実側面障害物距離 \verb|>| 果実アプローチ幅
  \item 果実側面障害物間距離が許容範囲内か: 果実側面障害物距離 \verb|>| 果実アプローチ幅
  \item 果実側面障害物間距離が許容範囲内か: 果実側面障害物距離 \verb|>| 果実アプローチ幅
  \item 果実アプローチ面積より果実が露出しているか: 果実露出面積 \verb|>| 果実アプローチ面積
\end{itemize}

% \begin{center}
%   \begin{tabular}{|lr|} \hline
%     花柄側面障害物距離が許容範囲内か & 花柄側面障害物距離 \verb|>| 花柄アプローチ幅\\ \hline
%     花柄アプローチ面積より花柄が露出しているか & 花柄露出面積 \verb|>| 花柄アプローチ面積\\ \hline
%     果実側面障害物距離が許容範囲内か & 果実側面障害物距離 \verb|>| 果実アプローチ幅\\ \hline
%     果実上面障害物距離が許容範囲内か & 果実上面障害物距離 \verb|>| 果実アプローチ幅\\ \hline
%     果実下面障害物距離が許容範囲内か & 果実下面障害物距離 \verb|>| 果実アプローチ幅\\ \hline
%     果実アプローチ面積より果実が露出しているか & 果実露出面積 \verb|>| 果実アプローチ面積\\ \hline
%   \end{tabular}
%   \end{center}