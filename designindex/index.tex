\section{設計指標}
収穫物の特性とエンドエフェクタの形状から設計指標を構築する.
各エンドエフェクタはそれぞれピーマンにアプローチする部分が異なるため, 必要に応じて指標を取り入れることになる.
例えば, 花柄アプローチ型のエンドエフェクタの場合は, 果実にはアプローチを行わないため, 花柄に関わる指標しか取り入れないこととなる.
\begin{center}
  \begin{tabular}{|lr|} \hline
    花柄側面障害物距離が許容範囲内か & 花柄側面障害物距離 \verb|>| 花柄アプローチ幅\\ \hline
    花柄アプローチ面積より花柄が露出しているか & 花柄露出面積 \verb|>| 花柄アプローチ面積\\ \hline
    果実側面障害物距離が許容範囲内か & 果実側面障害物距離 \verb|>| 果実アプローチ幅\\ \hline
    果実上面障害物距離が許容範囲内か & 果実上面障害物距離 \verb|>| 果実アプローチ幅\\ \hline
    果実下面障害物距離が許容範囲内か & 果実下面障害物距離 \verb|>| 果実アプローチ幅\\ \hline
    果実アプローチ面積より果実が露出しているか & 果実露出面積 \verb|>| 果実アプローチ面積\\ \hline
  \end{tabular}
  \end{center}