%!TEX root = ../thesis.tex
\chapter*{概要}
\thispagestyle{empty}
%
\begin{center}
  \scalebox{1.5}{収穫ロボット用エンドエフェクタの開発}\\
  \scalebox{1.5}{- 設計指標に関する検討 -}
\end{center}
\vspace{1.0zh}
%

近年, 農業従事者の減少と高齢化による問題を解決をするために収穫ロボットの開発が進められており, 収穫ロボット用のエンドエフェクタも既に様々な形状のものが提案されている.
しかし, これらの設計指針は明確に示されておらず, 収穫物に対して適切な形がわからないことや具体的な改善点を見つけにくいという問題がある.
そのため, 本研究ではいくつかの収穫物の特性を定量化し, 設計指標を構築した.
果実の収穫は複雑なため, 正面からの果実のアプローチに限定している.
また, 果実は多種多様なため, 今回はピーマンを対象に議論している.
設計指標の有効性を検証するために, 3Dスキャナでピーマンを3Dモデル化して必要な特性を計測した.
さらに, 既に提案されているピーマン収穫用のエンドエフェクタを参考にしてエンドエフェクタのモデルを3つ作成し, スキャンしたピーマン30個を対象にアプローチさせた.
作成したモデルはグリッパ型, 振動ナイフ型, 花柄アプローチ型の3つである.
そして, 計測した特性と作成したエンドエフェクタモデルの形状から成功率を推定し, 実環境でのアプローチの成功率と比較した. 
アプローチの成功数は異なるものの成功率の傾向は同じであり, 花柄アプローチ型, グリッパ型, 振動ナイフ型の順に成功率が高いことを確認した.


キーワード: 収穫ロボット, エンドエフェクタ, 3Dスキャナ
%
\newpage
%%
\chapter*{abstract}
\thispagestyle{empty}
%
\begin{center}
  \scalebox{1.3}{Development of an end effector for a harvesting robot}
  \scalebox{1.3}{- Consideration of design indicators -}
\end{center}
\vspace{1.0zh}
%

In recent years, harvesting robots have been developed to solve the problems caused by the decrease and aging of agricultural workers, and various shapes of end-effectors for harvesting robots have already been proposed.
However, the design guidelines for these end-effectors are not clearly defined, and it is difficult to find the appropriate shape for the harvested object or to find specific improvements.
Therefore, in this study, we quantified the characteristics of several harvested products and developed design indicators.
Due to the complexity of fruit harvesting, we have limited ourselves to a frontal approach to fruit.
In addition, because of the wide variety of fruits, we discuss only bell peppers in this study.
In order to validate the effectiveness of the design indicators, a 3D scanner was used to make a 3D model of a bell pepper and measure the necessary characteristics.
In addition, three models of end-effectors were created with reference to the end-effectors already proposed for harvesting green peppers, and 30 scanned green peppers were subjected to the approach.
The three models are a gripper type, a vibrating knife type, and a peduncle approach type.
The success rates were estimated from the measured characteristics and the shapes of the models, and compared with the success rates of the approaches in the real environment. 
Although the number of successful approaches differs, the trend of the success rate is the same, and it is confirmed that the success rate is higher for the peduncle approach type, the gripper type, and the vibrating knife type, in that order.

keywords: Harvesting robot, End effector, 3D scanner
