\section{実験目的}
構築した設計指標をもとに成功率を算出するため, 収穫物を計測し, 必要な特性のデータを取得する.
実環境のピーマンを手作業で計測していくのは困難なため, ピーマン株の3Dスキャンを行い, 3Dモデル化したピーマンをBlender内で計測する.
詳しくは次章で説明するが, スキャンしたピーマンに対して実際にエンドエフェクタのモデルをアプローチさせるため, 人が視認できるピーマン30個にシールを貼り, ID付けを行った.
