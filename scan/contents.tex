\section{実験方法}
収穫物の特性を計測する手順は以下のとおりである.
計測する特性は3.1節で提案した特性と, アプローチ面積を求めるために花柄と果実の幅をそれぞれ計測する.
計測対象は30個のピーマンだが, 奥の方にあるピーマンはスキャンできていないことや3Dモデルが欠損している場合は計測が困難なため, 計測結果からは除外する.
\begin{enumerate}
  \item 左から順に人が視認できるピーマン30個にID付け用のシールを貼る
  \item シールを貼ったピーマンがあるピーマン株に対して, Artec Leoを用いて3Dスキャンを行う
  \item 計測して得たピーマン株の3DモデルをBlender内に取り込み, 特性を計測する
\end{enumerate}